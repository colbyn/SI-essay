%%%%%%%%%%%%%%%%%%%%%%%%%%%%%%%%%%%%%%%%%%%%%%%%%%%%%%%%%%%%%%%%%%%%%%%%%%%%%%%
% TODO: REMOVE; THE FOLLOWING IS FROM AN OLD VERSION
%%%%%%%%%%%%%%%%%%%%%%%%%%%%%%%%%%%%%%%%%%%%%%%%%%%%%%%%%%%%%%%%%%%%%%%%%%%%%%%

\section*{Introduction}

My paper is on the essay by Larry Gordon, ``Wikipedia Pops Up in Bibliographies and Even College Assignments''.

\section*{Wikipedia Pops Up in Bibliographies and Even College Assignments}

Overall the essay is rather neutral, to me it was documenting a phenomena where some universities have began to integrate Wikipedia into their coursework. As the author wrote,

\begin{quotation}
``Once the bane of teachers, Wikipedia and entry-writing exercises are becoming more common on college campuses as academia and the online site drop mutual suspicions and seek to cooperate. In at least 150 courses at colleges in the U.S. and Canada, including UC Berkeley, UC San Francisco’s medical school, Boston College and Carnegie Mellon University, students were assigned to create or expand Wikipedia entries this year.''
\end{quotation}

Honestly this was rather surprising. I began this semester with the preconception that Wikipedia was looked upon with distaste. So you can imagine my surprise when I read of this development. The author even remarked that when Wikipedia began,

\begin{quotation}
“even as its popularity soared among the public, Wikipedia earned a reputation among academics as amateurish, peppered with errors and too open to nasty online spats over content. Wikipedia has tried to repair all that with better safeguards and a wider range of topics.”
\end{quotation}


Yet contrary to the remark that Wikipedia is “peppered with errors and too open to nasty online spats over content”, the author then recounted a story that contributing to Wikipedia was in fact rather difficult for a given class. As the author wrote,

\begin{quotation}
“Turns out it was a lot harder than the students anticipated. Their projects had to be researched, composed and coded to match Wikipedia’s strict protocols. Schug and her classmates wound up citing 218 scholarly legal and newspaper sources for their entry on a 1978 U.S. Supreme Court decision allowing corporate donations for ballot initiative campaigns.” 
\end{quotation}


Which given the subsequent remark that “Even the best research papers get buried in a drawer somewhere”, this to me is great news! It’s anything it’s unfortunate that quality work will die following the submission of the assignment. While contributing to wikipedia seems like a much more productive and far-reaching excursive, and something I wish was more commonplace.

As though things are changing, contrary to the “bane of teachers” remark, the author thereafter wrote,
\begin{quotation}
``Further symbolizing peace with academia, professional scholarly organizations in sociology, psychological science and communications in recent years have urged members to write Wikipedia articles and to assign students to do so.''
\end{quotation}

Likewise, the author again noted an incident with one student where,

\begin{quotation}
``In revising and broadening the entry on strokes, medical student Andrew Callen experienced Wikipedia’s argumentative nature. A Wikipedia medical editor, apparently a physician, challenged some of Callen’s technical terminology.  Callen said his language was more precise but conceded after some back and forth that the distinction was not important for lay readers.''
\end{quotation}

An interesting remark. Essentially, a Wikipedia editor has even begun optimizing content for laypersons. If anything this signifies something about Wikipedia, as if Wikipedia has begun a transition from struggling with factual information to curating content for the layperson. It makes me wonder how common this is throughout the community. 

Though, as if to diminish the former points, the author later wrote, “Some skepticism remains”, that follows, “Doug Hesse, vice president of the National Council of Teachers of English, said Wikipedia’s understandable insistence on neutrality doesn’t allow students to make reasoned arguments and analysis in term papers.” Personally, I’m not sure what the author meant by this. Perhaps he was referring to assignments that don’t just expect students to merely document the facts, but define their statement as it pertains to such facts, which obviously isn’t acceptable on Wikipedia. So such criticism seems reasonable (pretty much what I’ve done in every assignment for this class). 

The author continued, “And its reliance on published sources eliminates students’ independent interviews, experiments and research, said Hesse, who heads the University of Denver’s writing program.” Which again seems reasonable. Though on the computer-science related subset, links to e.g. open source projects are numerous. Personally, I think that if the students research contains substantive material, it seems reasonable to document such on Wikipedia (if it’s accepted by the editors), and if anything, contributions by the authors themselves seems preferable. 

Which, leading to the conclusion of the paper, the author remarked,

\begin{quotation}
“At Carnegie Mellon in Pittsburgh, professor of human-computer interaction Robert Kraut has assigned classes to compose Wikipedia chapters in psychology. Students have benefited, he said, but he, too, doesn’t think such assignments will become commonplace. Compared to regular term papers, Wiki entries require a lot more faculty time to ensure they are ready for online viewing. Some colleges may be put off by the public editing, which Kraut said led to some of his students’ writings getting excised for not following what he considered to be very complicated footnoting rules.”
\end{quotation}

Which the author ultimately concluded, ‘Freshman Lane Miles, who worked on the FairVote research, said it was doubly satisfying to help build the online encyclopedia. “We are educating ourselves and educating others,” he said.’

\section*{Philosophical Remarks}

While our traditional publication system is guarded by gatekeepers, the openness of Wikipedia entails an immune system that evolves alongside foreign attackers. If anything the fact that Wikipedia hasn’t imploded already from such pressures is an amazement. 

But more generally, to me, I think part of the problem is that fundamentally, Wikipedia can’t and doesn’t play by our preexisting standards in publication, and furthermore perhaps we shouldn’t hold it accountable to our preferred standards in publication. 

Obviously there are disadvantages to crowdsourcing content. Yet I personally believe that it’s advantages outweigh such. One such advantage is that fundamentally (based on personal experience) only Wikipedia can keep pace with the information generated nowadays. Because e.g. no mainstream publication will include anything akin to the niche content that Wikipedia has documented, and documented within a central, convenient, and -obviously- highly accessible platform. For instance, imagine you are developing a programming language for environments that don’t support or forbid higher order functions (i.e. Apple’s Metal API), and so need to reference a compilation pass called defunctionalization. Googling such (at the time of this writing) simply yields Wikipedia, or more ‘scholarly’ literature.

Furthermore regarding the present or past distaste for Wikipedia, the author even remarked, ’Even with complaints of mistakes and incompleteness, Wikipedia has a powerful reach. Often the first site suggested by Google searches, it has about 4.5 million English-language entries and 496 million visitors a month worldwide. Wikipedia “has essentially become too large to ignore,” said Berkeley’s 15 Kevin Gorman’.


\section*{Criticism}

As I said in the beginning, the overall style of the essay was rather neutral, more informal if anything. Which isn’t necessarily bad, yet he never addressed how wikipedia works in contrast to the traditional publication process, and the straights and weaknesses of such. In this context, he never mentioned how Wikipedia articles generally ‘evolve’ over time, from low quality, to high quality material. The author mentioned that a common criticism of Wikipedia is that content may be incomplete, but this is to be expected.

Ultimately -I think- what matters is that articles evolve over time, and due to this evolution the beginning will never be perfect, but it exists, and I believe this is important. Again, only VIA crowdsourcing content can we hope to document the ever growing volume of information generated nowadays.

