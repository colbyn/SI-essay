\section{Commentary}



\subsection{Preconceptions}

The author begins with a -perhaps- relatable story: in high school, ``Ani Schug was told to steer clear of Wikipedia.''; in College, the author continued, ``her American politics professor at Pomona College assigned the class to write detailed entries for Wikipedia instead of traditional term papers.''. So the dichotomy between Wikipedia and academia is introduced, but with an interesting twist. Regarding the latter, the author recounted that participating students experienced some unique challenges, since, ``[the] projects had to be researched, composed and coded to match Wikipedia’s strict protocols''.



\subsection{Ongoing Developments}

Furthermore, the author claims that this dichotomy between Wikipedia and academia is improving, claiming that,

``In at least 150 courses at colleges in the U.S. and Canada, including UC Berkeley, UC San Francisco’s medical school, Boston College and Carnegie Mellon University, students were assigned to create or expand Wikipedia entries this year.''

The author continued,

``The result, supporters say, has been better researched articles about, for example, the causes of paralyzing strokes and the history of the American West. And, they say, students are becoming better prepared for a future of digital information.''

This is important because as Amanda Hollis-Brusky put it, ``Even the best research papers get buried in a drawer somewhere'', whereas the aforementioned assignments ``make a real contribution to the public discourse''.



\subsection{Acknowledging Reality}

The author claims that regardless of ``complaints of mistakes and incompleteness, Wikipedia has a powerful reach.'' Given that, for instance, ``Often the first site suggested by Google searches, it has about 4.5 million English-language entries and 496 million visitors a month worldwide.''

So therefore, as Berkeley’s Kevin Gorman put it,

``[Wikipedia] has essentially become too large to ignore, [...] It is certainly an initial source of information for a huge number of people, [...] For many people, it may be their primary source of information.''. 


\subsection{Tales of Mutual Benefit and Corporation}

The author remarks, "Further symbolizing peace with academia" a multitude of ``scholarly organizations'' have begin to ``urged members to write Wikipedia articles'', and of course, ``to [likewise] assign students to do so''.

`In the first such class at an American medical school, students have started or revised pages about hepatitis, dementia and alcohol withdrawal syndrome, among others, Azzam said. [..] The assignments, he explained, are part of young doctors’ “social contract to do good in the world and help patients” learn about health.'

Even though such contributions will still need to go through some peer review process, according to Wikipedia's rules. For instance, the author recounted a story where a medical student's proposed edition was rejected. Stating that, while the ``language was more precise'', the editor said, ultimately the content ``was not important for lay readers''. Nevertheless, the student remarked that such exercises had utility for both the students, and obviously Wikipedia, to which the student remarked, ``The more people we can get to edit it, the more accurate the information will be''.


\subsection{New Skepticism?}

In the context of Wikipedia based assignments, the author noted that Doug Hesse, who heads the University of Denver’s writing program said:

``Wikipedia’s understandable insistence on neutrality doesn’t allow students to make reasoned arguments and analysis in term papers. [..] its reliance on published sources eliminates students’ independent interviews, experiments and research''.

Another instance was given from Robert Kraut from Carnegie Mellon, who said, ``[while] students have benefited'', he ``doesn’t think such assignments will become commonplace''.

Likewise, given the nature of Wikipedia based assignments, the author noted that such requires, ``more faculty time to ensure they are ready for online viewing''.



